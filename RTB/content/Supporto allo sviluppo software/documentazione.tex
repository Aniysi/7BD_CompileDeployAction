\subsection{Documentazione}

\subsubsection{Scopo}
La documentazione è l'insieme dei contenuti che accompagnano la creazione 
del prodotto software, e svolge un ruolo essenziale nella descrizione 
dello stesso e nella definizione dei processi che portano alla sua realizzazione.
Il fine della documentazione è quello di facilitare il lavoro dei componenti del
team software, e guidarli nella realizzazione del prodotto. Questo mira a

\subsubsection{Lista documenti}
I documenti prodotti nel contesto della realizzazione del progetto sono:
\begin{itemize}
    \item \textit{Analisi\_dei\_requisiti.pdf}
    \item \textit{Glossario.pdf}
    \item \textit{Norme\_di\_progetto.pdf}
    \item \textit{Piano\_di\_progetto.pdf}
    \item \textit{Piano\_di\_qualifica.pdf}
    \item Verbali esterni
    \item Verbali interni
\end{itemize}

\subsubsection{Ciclo di vita documenti}
I documenti seguono due workflow distinti a seconda che si tratti di verbali 
(interni o esterni) oppure di documenti più corposi, e le versioni delle modifche 
successive apportate ai documenti seguono il sistema di versionamento indicato di seguito.\\

    \paragraph{Versionamento dei documenti}
    Il sistema adottato dal team per il versionamento dei documenti è il sistema di
    \textbf{versionamento semantico}: \textbf{x.y.z}, in cui ogni numero (x, y, z) 
    ha un significato specifico: x indica la versione maggiore (major), incrementata 
    per cambiamenti incompatibili con versioni precedenti; y rappresenta la versione 
    minore (minor), usata per aggiungere informazioni compatibili; z è la versione di 
    patch, aggiornata per correzioni di errori poco significativi e retrocompatibili.
    Questo sistema aiuta il team a comprendere velocemente l'impatto di un aggiornamento
    fatto a un documento prodotto.

    \paragraph{Workflow verbali}
    I verbali seguono il seguente workflow:
    \begin{enumerate}
        \item Creazione del documento a partire da un template comune
        a tutti i verbali (la versione iniziale del documento è già 
        definita in tale template e corrisponde a \textit{0.1.0});
        \item Compilazione dei campi della sezione di \textit{Registro 
        delle modifiche};
        \item Redazione del documento indicando partecipanti (interni e 
        esterni), la sintesi di quanto fatto e una descrizione di ciascuna 
        delle considerazioni fatte e successive decisioni prese;
        \item Nella sezione di \textit{Decisioni prese} compilazione della 
        tabella in cui ciascuna azione da intraprendere viene associata 
        a una issue corrispondente;
        \item Creazione di una pull request dal branch \framebox{Verbali}
        al branch \framebox{Main};
        \item Verifica del verbale prodotto da parte del verificatore 
        indicato nel \textit{Registro delle modifiche} del documento stesso;
        \item Se ci sono correzioni o ulteriori modifiche da fare, queste 
        devono essere indicate a loro volta nella sezione 
        \textit{Registro delle modifiche} per poi essere verificate;
        \item Quando il documento è completo l'ultimo verificatore 
        chiude la pull request e esegue il merge nel branch \framebox{Main};
    \end{enumerate}

    \paragraph{Workflow altri documenti}
    Gli altri documenti seguono il seguente workflow:
    \begin{enumerate}
        \item Creazione del documento a partire da un template comune 
        suddiviso in file \textit{.tex} distinti (uno per ogni componente del 
        documento finale). La versione iniziale del documento è già 
        definita in tale template e corrisponde a \textit{0.1.0};
        \item Creazione di una draft pull request dal branch corrispondente 
        a tale documento al branch \framebox{Main};
        \item Compilazione dei campi della sezione di \textit{Registro 
        delle modifiche};
        \item Redazione del documento o di alcune delle sue parti;
        \item Verifica della doumentazione prodotta da parte del verificatore 
        indicato nel \textit{Registro delle modifiche} e associato alle modifche 
        effettuate in quella seduta di lavoro;
        \item Se ci sono correzioni o ulteriori modifiche da fare, queste 
        devono essere indicate a loro volta nella sezione 
        \textit{Registro delle modifiche} per poi essere verificate;
        \item Si ripetono le operazioni indicate dal punto \textit{3} al punto 
        \textit{6} fino a quando il documento non è stato completato;
        \item Quando il documento è completo l'ultimo verificatore 
        chiude la draft pull request e esegue il merge nel branch \framebox{Main};
    \end{enumerate}
