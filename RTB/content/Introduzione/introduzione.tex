\section{Introduzione}
    \subsection{Funzione del documento}
    Il presente documeto è stato realizzato ai fini di definire e reccagliore le best 
    practices e il way of working a cui ogni componente del gruppo Seven Bits dovrà
    aderire per l'intera realizzazione del progetto, al fine di garantire l'adozione 
    di un metodo di lavoro completamente omogeno\\
    La formulaziome delle norme di porgetto avviene in maniera progressiva, permettendo
    al gruppo di apportare continui aggiornamenti ad esse in risposta alle esigenze che
    il team deve affrontare durante lo svolgimento del progetto stesso.\\ 

    \subsection{Glossario}
    Ai fini di garantire l'adesione dei membri del gruppo a un vocabolario comune 
    e condiviso, che non lasci spazio a ambiguità, dubbi o imprecisioni; il team ha definito
    un documento denominato Glossario, nel quale sono presenti tutti i termini tecnici adottati
    dal gruppo per l'intera durata della realizzazione del progetto.

    \subsection{Riferimenti}
        \subsubsection{Riferimenti progettuali}
        \begin{itemize}
            \item \href{https://www.math.unipd.it/~tullio/IS-1/2024/Progetto/C4.pdf}{Descrizione capitolato di porgetto}
            \item Presentazione capitolato di progetto: 
        \end{itemize}
        \subsubsection{Riferimenti tecnologici}
        \begin{itemize}
            \item \href{https://git-scm.com/docs}{Documentazione git}
            \item \href{https://docs.github.com/en}{Documentazione GitHub}
            \item \href{https://www.latex-project.org/help/documentation/}{Documentazione LATEX}
            \item \href{https://www.python.org/doc/}{Documentazione Python:}
        \end{itemize}