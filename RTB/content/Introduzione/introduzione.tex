\section{Introduzione}
    \subsection{Scopo del documento}
    Il presente documento è stato realizzato ai fini di definire e raccogliere le best practices e il way of working a cui ogni componente del gruppo Seven Bits dovrà aderire per l'intera realizzazione del progetto, al fine di garantire l'adozione di un metodo di lavoro completamente omogeno.\\
    La formulaziome delle norme di progetto avviene in maniera progressiva, permettendo al gruppo di apportare continui aggiornamenti ad esse in risposta alle esigenze che il team dovrà affrontare durante lo svolgimento del progetto stesso.\\ 

    \subsection{Scopo del prodotto}
    Ogni giorno, le persone vengono sommerse da una miriade di annunci generici che spesso non rispecchiano i loro reali interessi o il contesto in cui si trovano. Questa separazione tra il messaggio e il destinatario porta ad una bassa interazione con gli utenti e una riduzione delle conversioni per i brand.\\
    Il progetto “Near You” si concentra sulla creazione di una dashboard composta principalmente da una mappa, sulla quale verranno visualizzate in tempo reale le posizioni degli utenti. Mediante un popup o una finestra a parte, verranno visualizzati messaggi personalizzati solo in prossimità dei punti di interesse.\\
    L'obiettivo finale è generare annunci pubblicitari in base agli interessi del cliente e alla sua posizione in quel momento.\\
    
    \subsection{Glossario}
    Ai fini di garantire l'adesione dei membri del gruppo ad un vocabolario comune e condiviso, che non lasci spazio ad ambiguità, dubbi o imprecisioni; il gruppo ha definito un documento denominato Glossario, nel quale sono presenti tutti i termini tecnici adottati dal gruppo per l'intera durata della realizzazione del progetto. Tali termini saranno evidenziati in \textit{corsivo} e contrassegnati con una $_G$ a pedice.

    \subsection{Riferimenti}
        \subsubsection{Riferimenti progettuali}
        \begin{itemize}
            \item \href{https://www.math.unipd.it/~tullio/IS-1/2024/Progetto/C4.pdf}{Capitolato di progetto C4 - Near You}
        \end{itemize}
        \subsubsection{Riferimenti tecnologici}
        \begin{itemize}
            \item \href{https://git-scm.com/docs}{Documentazione git}
            \item \href{https://docs.github.com/en}{Documentazione GitHub}
            \item \href{https://www.latex-project.org/help/documentation/}{Documentazione \textit{LaTeX}$_G$}
            \item \href{https://www.python.org/doc/}{Documentazione Python}
        \end{itemize}